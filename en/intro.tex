\section{Introduction}

Generally, developers develop applications on some application platforms (like
Windows\footnote{\url{https://www.microsoft.com/en-us/windows}},Linux\footnote{\url{https://en.wikipedia.org/wiki/Linux}},
macOS\footnote{\url{https://en.wikipedia.org/wiki/MacOS}},
iOS\footnote{\url{https://en.wikipedia.org/wiki/IOS}},
Android\footnote{\url{https://en.wikipedia.org/wiki/Android}} etc.), and
benefit from their applications in traditional software development industry.
The way to get the benefits varies for different developers, including but not
limited to salaries paid by software enterprises, revenue by selling the
applications or displaying advertises in their applications.

Yet, the enterprises who build the application platforms also benefit from
applications, and such benefits are not shared with those application
developers. Take operating systems as examples since they are typical
application platforms. Say a designer needs to use \texttt{Sketch}, and he/she has to
buy a \texttt{macOS} device, which means he/she has to pay
Apple\footnote{\url{https://en.wikipedia.org/wiki/Apple_Inc.}} besides paying
Sketch because Sketch requires
macOS\footnote{\url{https://www.sketchapp.com/support/requirements/other-platforms/}}.
Apple benefits from such users. However, Apple will not share the benefit with
Sketch developers. Another similar example is that users have to pay Apple or
Microsoft\footnote{\url{https://en.wikipedia.org/wiki/Microsoft}} to use
AutoCAD\footnote{\url{https://en.wikipedia.org/wiki/AutoCAD}}. In such cases,
the key factor of choosing a platform is whether the platform
supports the needed applications for users. In other words, quality
applications are critical to the growth of an application platform. Thus, we
believe that application developers should be guaranteed to share the profits
of application platforms.

The same thing, that DApp (decentralized application) developers don't share
the profits, happens in blockchain industry. In 2004, Ethereum community
proposed smart contract, which extended blockchains' ability from peer-to-peer
cryptocurrency networks to decentralized application platforms. However, it is
quite similar between decentralized application development and traditional
application development. And decentralized application developers still benefit
nothing from a blockchain's increased value.


In the abstract, the increased value of a blockchain comes from the new block
reward. We believe that a blockchain's value comes from users' data. Thus, the
new block reward should be distributed to all parties that contributes the
new data. DApp developers are key parts of the contributors, and they should
share the new block reward. However, Bitcoin and such PoW (proof
of work) systems distribute the new block reward
to the miners, while PoS (proof of stake) blockchain systems distribute the new
block reward to the stake holders. Consequently, DApp developers benefit
nothing.

Conceptually, a DApp is a set of smart contracts that for specific
functionality, while a smart contract is a computer protocol intended to
digitally facilitate, verify, or enforce the negotiation or performance of a
contract. Smart contracts allow the performance of credible transactions
without third
parties\footnote{\url{https://en.wikipedia.org/wiki/Smart\_contract}}.
Currently, most DApps use smart contracts as backend, and applications on PC,
mobile or web as frontend.

We believe that blockchain systems or DApp platforms, DApp developers and DApp
users are closely related with each other. First, DApp platforms provide
necessary facilities for DApp developers so that more and more DApp developers
can build their own DApps and benefit from the DApps. Second, DApp
developers build DApps for different scenarios, which bring more and more
users for the DApp platforms. Finally, DApp users produce onchain data,
increase liquidity, and increase the value of a blockchain system.

Notice that we mean DApp developers instead of Nebulas DApp developers or
blockchain system developers. Thus, we shall use developers short for DApp
developers without ambiguity. Also, a DApp developer may be a stake holder.
He/she may benefit from being a stake holder and that benefit is not considered
as sharing the increased value of blockchains.


It's difficult to justly distribute a platform's increased value to
corresponding application developers. First, the revenue is owned by
centralized organizations, like an enterprise, and application developers have
no chance to know the details or participate in sharing the revenue. Second, it
is difficult to quantify each developer's contribution to the growth of
application platforms. Fortunately, this situation can be changed in blockchain
industry since each invocation for smart contracts by each user is
publicly recorded on
blockchain. Thus, it is possible to \emph{reward or incentive each DApp developer by
quantifying each DApp's contribution}.

An ideal incentive mechanism should satisfy some basic properties:
\begin{itemize}
\item Justic:
\item Effectiveness:
\end{itemize}


