\section{Introduccion}

\noindent
Generalmente, cuando un desarrollador de software crea aplicaciones dentro de las plataformas más conocidas (como
Windows\footnote{\url{https://www.microsoft.com/es-es/windows}}, Linux\footnote{\url{https://es.wikipedia.org/wiki/Linux}},
macOS\footnote{\url{https://es.wikipedia.org/wiki/MacOS}},
iOS\footnote{\url{https://es.wikipedia.org/wiki/IOS}},
Android\footnote{\url{https://es.wikipedia.org/wiki/Android}} \etc) lo hace buscando un beneficio económico en el sentido tradicional de la industria del software: mediante salarios pagados por las empresas para las cuales trabajan en relación de dependencia, mediante ingresos por licencias de aplicación, o bien mediante anuncios que se muestran dentro de sus aplicaciones. A su vez, las empresas que contratan a esos desarrolladores reciben ingresos —por la venta del software desarrollado por ellos—, que no comparten con sus trabajadores.

Veamos un ejemplo basado en el desarrollo bajo la plataforma macOS: un diseñador UI/UX que necesita hacer uso de \textit{Sketch} debe pagar no sólo por los complementos de esa herramienta, sino que además debe pagarle a Apple\footnote{\url{https://en.wikipedia.org/wiki/Apple_Inc.}}
por el dispositivo necesario para correr la herramienta. Así, Apple se beneficia de ese \textit{cliente cautivo} pero no comparte esos ingresos con los desarrolladores de la herramienta \textit{Sketch}. Algo similar ocurre con la popular herramienta AutoCAD\footnote{\url{https://en.wikipedia.org/wiki/AutoCAD}}, por la cual es necesario pagar a Microsoft\footnote{\url{https://en.wikipedia.org/wiki/Microsoft}} o a Apple el sistema operativo necesario para poder utilizarla. En estos casos, el factor clave para que los usuarios elijan una plataforma es si la plataforma soporta las aplicaciones que los usuarios necesitan. Es decir, las aplicaciones de alta calidad son fundamentales para el desarrollo de una \textit{plataforma de aplicaciones}.

Basándonos en las consideraciones anteriores, podemos decir que —hasta cierto punto— las plataformas de aplicaciones no toman en cuenta las necesidades de los desarrolladores, dañando así sus intereses.

Este escenario se replica también en la industria blockchain, en donde los desarrolladores de aplicaciones descentralizadas (DApps) están siendo ignorados por las plataformas para las cuales desarrollan. En 2014, la comunidad de Ethereum introdujo el concepto de \textit{contratos inteligentes}, que extienden la capacidad del \textit{blockchain} de una simple herramienta de registro para criptodivisas p2p a una verdadera plataforma de aplicaciones descentralizadas. Sin embargo, a semejanza de la industria centralizada del desarrollo de software, no hubo mejoras significativas en las ganancias para los desarrolladores —entre otras cosas, porque no tienen forma de beneficiarse de la apreciación en el valor de las plataformas blockchain para las cuales desarrollan.

De forma general, las recompensas emitidas por cada nuevo bloque en un \textit{blockchain} representan un incremento de valor para ese sistema, y la distribución de tales recompensas determina el camino de los incentivos en esos sistemas descentralizados. En nuestra opinión, el incremento del valor de un sistema \textit{blockchain} proviene esencialmente del valor implícito de los datos de sus usuarios, algo que debería distribuirse entre todas las partes intervinientes, incluyendo a los desarrolladores de aplicaciones descentralizadas. Aun así, lo que se ve en la práctica, al menos en la mayoría de los blockchains basados en PoW\footnote{Prueba de Trabajo, o \textit{Proof of Work} en inglés} —representados por Bitcoin y sus clones— es que las recompensas por nuevos bloques se distribuyen únicamente entre mineros; en los blockchains basados en el algoritmo PoS\footnote{Prueba de Participación, o \textit{Proof of Stake} en inglés}, las recompensas se asignan mayormente a quienes son propietarios de grandes sumas de la criptodivisa asociada. Así, los intereses de los desarrolladores de aplicaciones descentralizadas se ven afectados.

Conceptualmente, una \dapp es un conjunto de contratos inteligentes con una serie de funcionalidades específicas; un contrato inteligente es, a su vez, un protocolo computacional dirigido a facilitar, verificar o forzar digitalmente la negociación o la realización de un contrato. Estos contratos inteligentes permiten la realización de transacciones fiables sin necesidad de la intervención de terceros\footnote{Cf. \url{https://es.wikipedia.org/wiki/Contrato\_inteligente}}.

Desde el punto de vista de la arquitectura de estos sistemas, la mayoría de las {\dapp}s utilizan usualmente contratos inteligentes en el \textit{backend}, mientras hacen uso de tecnologías más tradicionales para el \textit{frontend}. Las {\dapp}s pueden presentarse como aplicaciones tradicionales de PC, como aplicaciones móviles o bien como aplicaciones web.

Creemos que la relación entre las plataformas de aplicaciones descentralizadas, sus desarrolladores y sus usuarios es sinérgica y simbiótica. En primer lugar, la emergencia de aplicaciones descentralizadas permitió la expansión de la comunidad de desarrolladores blockchain; existe un crecimiento sostenido en la cantidad de desarrolladores que intentan escribir {\dapp}s que cumplan distintos requerimientos y que les permitan beneficiarse económicamente de ellas. En segundo lugar, esos desarrolladores proveen riqueza a las {\dapp}s, expandiendo así los escenarios posibles de uso para los blockchains, y a su vez atraen más usuarios hacia esas plataformas. Finalmente, los usuarios de {\dapp}s son quienes guían la optimización y actualización permanentes de estas plataformas, incrementando en paralelo la movilidad de los tokens en las plataformas, y logrando en conjunto el crecimiento y el desarrollo del sistema blockchain.

Es necesario notar que los desarrolladores descriptos aquí sólo refieren a quienes escriben aplicaciones para plataformas descentralizadas, no necesariamente desarrolladores para la plataforma Nebulas ni tampoco necesariamente desarrolladores dentro del sistema blockchain (por esa razón hacemos mención de \textit{desarrolladores de DApps} en forma genérica). Así, en este documento usaremos el término \textit{desarrolladores} como sinónimo para \textit{desarrolladores de {\dapp}s} con el fin de evitar esa ambigüedad. También es importante notar que un desarrollador \dapp puede ser también un \textit{stake holder} que obtiene beneficios económicos por esa vía; sin embargo, debido a que no todos los desarrolladores tienen asegurado ese derecho ni esa posesión, no es posible afirmar que el \textit{stake holding} sea un medio legítimo para el incremento del valor de un blockchain; los intereses del desarrollador pueden ser ignorados o vulnerados sin perjuicio de este hecho.

Es inapropiada la distribución de ganancias por el incremento de valor de una plataforma entre sus desarrolladores de aplicaciones; por un lado, los ingresos pertenecen a corporaciones centralizadas, y los desarrolladores de aplicaciones para esas plataformas no tienen forma de conocer los detalles de los ingresos, ni tampoco participar en la distribución de esas ganancias. Por otro lado, es difícil cuantificar la contribución de cada desarrollador al crecimiento de una plataforma dada, por lo que la equidad de un mecanismo de distribución tal sería cuanto menos difícil de garantizar.

Afortunadamente, esa situación se puede cambiar en la industria del blockchain, ya que cada llamada a un contrato inteligente por parte de los usuarios queda registrada de forma pública en el blockchain. Así, \emph{es posible otorgar recompensas o incentivos a cada desarrollador de {\dapp}s simplemente mediante la cuantificación de cada contribución de una {\dapp} dada}.

Un mecanismo ideal para la distribución de incentivos debe satisfacer algunas propiedades básicas:

\begin{itemize}
	\item Equidad: el protocolo debe mantener la objetividad al recompensar a los desarrolladores; esto es, cada {\dapp} debe tratarse equitativamente y sus usos deben ser evaluados de forma transparente y verificable. Aun así existe cierto margen para manipulaciones.

	\item Efectividad: las recompensas deben reflejar las preferencias de los usuarios; esto es, aquellas {\dapp}s que reciben más recompensas deben coincidir con las que son de uso más frecuente por parte de los usuarios, mientras las {\dapp}s con recompensas bajas o nulas deben coincidir con aquellas que los usuarios evitan utilizar.
\end{itemize}

En este documento proponemos la implementación del Protocolo de Incentivos para Desarrolladores (\textit{Developer Incentive Protocol}, o DIP), que apunta a otorgar recompensas e incentivos a los desarrolladores, habilitándolos a beneficiarse del desarrollo de nuestra plataforma de aplicaciones descentralizadas. Naturalmente, no existe un protocolo de incentivos para desarrolladores ideal debido a que la evaluación de los usuarios sobre las  {\dapp}s son subjetivas y multidimensionadas. Así, el protocolo DIP presentado en este documento tiene todavía margen para su mejora. No obstante ello, el balance que deja este libro púrpura es innovativo, en el sentido de que la premisa es garantizar los intereses de los desarrolladores de {\dapp}s en términos de resistencia a la manipulación.

El diseño de DIP está basado en el sistema pre-existente \textit{Nebulas Rank}
(NR)\cite{Nebulasyellowpaper} y se beneficia de algunas cualidades del mismo. Intuitivamente, la evaluación de las DApps se reduce a un proceso de votación en el sistema DIP: cada llamada a una DApp por parte de un usuario se trata como un voto, y la capacidad de voto de cada usuario es una función de su número NR\footnote{Número adimensional que indica la valuación dada a ese usuario por el algoritmo Nebulas Rank}. Los desarrolladores obtendrán las recompensas por parte del sistema, eventualmente, de acuerdo a los resultados de la votación.

Más allá del análisis teórico del modelo del Protocolo de Incentivos para Desarrolladores, analizaremos también las medidas contra las posibles manipulaciones e ilustraremos la implementación del protocolo mencionado con ejemplos tales como el ajuste y la actualización de DIP\@.

\whitepaper{
Nota: este documento se enfoca en la discusión del Protocolo de Incentivos para Desarrolladores; a raíz de ello, la información aquí presente es más avanzada y actualizada que la disponible en el \textit{Libro Blanco Técnico} ~\cite{Nebulaswhitepaper}, cuya versión 1.02 se lanzó en abril de 2018. Comparado con la demostración conceptual llevada a cabo entonces, y luego de un año de verificaciones en la práctica, tenemos la confianza y la experiencia para crear algoritmos más rigurosos y soluciones más claras para este protocolo.
}