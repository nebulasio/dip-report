\section{概要}

一般而言,在传统的软件开发行业,开发者在某开发平台
(如Windows\footnote{\url{https://www.microsoft.com/en-us/windows}},Linux\footnote{\url{https://en.wikipedia.org/wiki/Linux}},
macOS\footnote{\url{https://en.wikipedia.org/wiki/MacOS}},
iOS\footnote{\url{https://en.wikipedia.org/wiki/IOS}},
Android\footnote{\url{https://en.wikipedia.org/wiki/Android}}等)上开发相应的应用,并由此获利。
开发者的获利方式主要包括,薪资、售卖自己主导或参与开发的软件以及在免费分发的软件中插入广告流量。

然而,应用平台方在软件的使用中所获得的利益并没有被公平的分配给相应的开发者。例如,某用户需要使用Sketch而不得不购买搭载macOS的Mac设备\footnote{\url{https://www.sketchapp.com/support/requirements/other-platforms/}},
并由此付出费用,尽管Sketch的开发者可以向用户收取使用费用,
但是通过售卖Mac设备而获利的Apple公司\footnote{\url{https://en.wikipedia.org/wiki/Apple_Inc.}}并不会因此额外向Sketch的开发者分享利益;类似的,某用户需要使用AutoCAD\footnote{\url{https://en.wikipedia.org/wiki/AutoCAD}}软件时,只能选择搭载macOS或Windows的设备,
由此需要支付给Apple或Microsoft\footnote{\url{https://en.wikipedia.org/wiki/Microsoft}}及相关企业
的费用亦并未向AutoCAD的开发者分享。
诚然, 用户选择一个平台的原因多种多样,其中一个关键性的影响因素是需要的应用所支持的平台,换句话说,一个应用平台的发展,离不开其上的优秀应用。
因此,应用平台方无视开发者利益的做法,一定程度上侵害了开发者的利益。

在区块链中,去中心化应用(Decentralized Application, DApp)开发者的利益同样被漠视。
2014年,以太坊提出在区块链上运行图灵完备的智能合约(Smart Contract),使得区块链从单纯的数字货币支付网络升级为了去中心化应用平台。
然而,DApp开发者的获利方式与传统的软件开发行业相比,并无明显区别,即,开发者并不能从去中心应用平台或区块链系统的发展中获利。

可以抽象的认为,区块链中出块奖励代表了区块链系统发展中新增的价值,而出块奖励的分发决定了去中心化系统的激励方向。我们认为,区块链系统增加的价值本质上来源于新增的用户数据所蕴含的价值,这些新增的价值
应该公平的分发给为系统的新增价值做出贡献的各方,其中,就包括DApp开发者。然而,在以比特币为代表的区块链系统中,出块奖励被发放给了矿工节点;在诸多的基于PoS(Proof of Stake)的区块链系统中,
出块奖励被发放给了系统代币的持有者;同样的,在诸多的区块链系统中,DApp开发者的利益一定程度上都被漠视或侵害了。

一般而言,一个去中心化应用可以理解为为了实现特定功能的一系列智能合约的集合。
智能合约是一种旨在以信息化方式传播、验证或执行合同的计算机协议。智能合约允许在没有第三方的情况下进行可信交易\footnote{\url{https://en.wikipedia.org/wiki/Smart\_contract}}。
从技术架构来看,大部分DApp通常以智能合约作为后端,同时采用了常见的前端技术与之交互,因此DApp形态既可以是传统个人电脑客户端,也可以是移动应用或者网页应用。

我们认为,去中心化应用平台、DApp开发者及DApp用户,这三者是互相促进的。
首先,去中心化应用平台的出现,扩大了区块链开发者这一群体,越来越多的开发者尝试开发满足不同需求的DApp,并从DApp的开发中获益;
其次,DApp开发者提供了丰富多样的DApp,扩大了区块链的应用场景,为区块链带来了更多的用户;
最后,DApp用户驱动着去中心化应用平台的不断的优化、升级,增加去中心化应用平台之上代币的流通性,是的整个区块链系统得以发展。
因此,更进一步的,我们认为,DApp开发者的利益应该得到公平的分配,并给予保证,这是区块链做为去中心化应用平台维持可持续发展的关键所在。

需要注意,本文所述的开发者仅指去中心化应用平台之上的DApp开发者,不特指星云链上的DApp开发者,亦不包括区块链系统本身的开发者,因此,如无歧义,下文所述的开发者皆指DApp开发者。
特别的,DApp开发者在身份上可能同时持有一定数量的代币,而DApp开发者因为持币而带来的任何收益都不能等同的认为享受到了区块链系统发展带来的增值,
其做为DApp开发者的利益依然可能是被漠视甚至侵害的。


让应用开发者在平台的发展中公平获益,在技术上并不容易实现。一方面,在传统的软件开发行业中,平台的发展及获利状况及由中心化的组织掌握,平台之上的应用开发者无从知晓或参与相应的利益分配;
另一方面,应用开发者为平台发展所做出的贡献难以量化,利益分配难以做到公平公正。
而在区块链为基础的去中心化应用平台上,这一状况有望得到改善。区块链的价值来源于代币的流通性,DApp的使用情况被公开的记录在区块链上,得益于此,\textbf{基于DApp的使用情况,
量化DApp的使用对一个区块链系统的发展所做出的贡献,并进一步给予开发者公平的激励},是必要且可行的。

理想情况下,对于DApp开发者的激励需要满足如下基本性质:
\begin{itemize}
\item 公平性:对于DApp开发者的激励,需要保证相对客观,即每个DApp需要被公平的对待,其使用情况需要被真实的衡量,并摒弃可能存在的操纵行为。
\item 有效性:对于DApp开发者的激励,需要真实反映用户的偏好,即获得激励高的DApp是活跃用户所喜欢且经常使用的,而获得激励低的DApp是鲜有用户问津的。
\begin{comment}
	\item 公开性:链上的DApp激励协议与传统的评奖方式最大的不同在于,所有评分的机制必须是完全公开的,
	且其中任何统计、计算、评选的过程都是全程可见的。
	这样就杜绝了传统中心化评奖暗箱操作的可能。同时也不会出现票数统计出错等情况。
	最后,根据评选结果分配奖励的过程也会保证被执行,奖励分配正如链上交易一样可被追溯。
	\item 有效性:这也是任何评选机制所要满足的基本性质。我们期望DApp评分能够真实反映用户的评价,
	即排名高的DApp是活跃用户所喜欢的且经常被调用的,而评分低的DApp是用户鲜有问津的。
	\item 抗作弊:对于任何排名算法,都需要解决各类作弊问题。对于DIP而言,主要存在两类作弊问题。
	\begin{enumerate}
		\item 女巫攻击:区块链技术的一个重大特点就是一个用户建立新的节点地址代价是很小的。所以一个用户有可能建立多个由他控制的地址,并将他们伪装成多个正常用户来参与评选。
		一个好的激励协议应当保证每个用户无法通过女巫攻击带来巨大额外收益。
		\item 收买:由于我们衡量DApp好坏的主要指标是活跃用户调用的次数,一个DApp开发者有可能收买大量用户让他们调用自己的DApp以提高自己的排名从而获得更多奖励。
		这种作弊方式原则上无法杜绝,但我们期望激励协议能够让此类收买需要付出的代价变得很高以减少其出现的概率。
	\end{enumerate}
\end{comment}
\end{itemize}


本文提出DApp开发者激励协议(Developer Incentive Protocol,DIP),试图给予开发者激励,让DApp开发者能够公平的在去中心化应用平台的发展中获益。
我们深知,用户对于DApp的真实评价是主观且多维的,一个理想的DApp开发者激励协议可能是不存在的,因此,本文给出的DApp开发者激励协议依然存在各种不足,
其中仍然包含了我们对于DApp开发者的偏好。然而,在抵抗操纵及保证开发者利益之间,本文所做出的权衡依旧是创新性的,即,在保证DApp开发者利益的前提下,在抵抗操作方面,做出了最大限度的努力。

本文提出的开发者激励协议基于已有的星云指数(Nebulas Rank)~\cite{Nabulasyellowpaper},由于星云指数的相关性质,开发者激励协议能够在一定程度上很好的保证上述性质。
直观的来说,开发者激励协议将用户对DApp的使用情况简化为用户根据自己的喜好对DApp进行投票的问题,用户的投票总数为该用户的星云指数,而对DApp的使用则为对DApp的投票,
最终根据投票结果,对开发者给予相应比例的激励。

本紫皮书在给出开发者激励协议的理论模型之外,还对其抵抗操纵的性质进行了分析,并对系统中如何实现开发者激励协议给出了必要的说明,例如,如何对开发者激励协议进行必要的调整及更新,
从而对开发者激励协议的实际落地给出了具体的工作方向。

\whitepaper{
特殊提示:
本开发者激励协议紫皮皮书作为专项讨论开发者激励协议的紫皮书,对星云技术白皮书(2018年4月发布的1.02版本)~\cite{Nabulas}中开发者激励协议相关章节进行了大幅度的升级和拓展。
相对于一年前的概念论证,经过一年的深入思考与实际验证,我们有信心和能力设计出更为严谨的算法,并对星云指数的更多实际细节问题提供明确的解决方案或方向。
为了方便阅读,我们将使用实线框高亮解释技术白皮书提及过、并且在本紫皮书中有升级的相关技术点。
}
