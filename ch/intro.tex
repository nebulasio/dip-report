\section{概要}

1. 区块链关于dapp的介绍 

区块链行业现已进入2.0时代,其代表为以太坊白皮书的发布\cite{buterin2014next},其中支持图灵完备的智能合约的引入使得区块链应用开发者高效快地开发顶层应用变为可能。具体而言,一个分布式应用(dapp)可以理解为为了实现特定功能的一系列智能合约的集合。正如广为人知的appstore上的应用一样,dapp可以包括游戏、赌博、社交等多种类型。相比于传统的app应用,dapp因嵌套于区块链技术故同时具有去中心化,安全性等特点。

迄今为止,在以太坊上已有超过xxx款dapp。同时,星云和eos上也分别有xxx和xxx dapp。可以预见,这些dapp的质量也参差不齐。那么,如何对大量的dapp进行排名以提供更好的用户体验,以及如何促进新的高质量的dapp被开发成为每条公链必须解决的问题。

2. 引入开发者激励协议的必要性(何谓激励?)

所谓激励是指通过特定的方法让用户产生行为动机。而开发者激励协议(DIP)是指以鼓励开发者开发优秀dapp为目的,以一套具有抗作弊性的dapp排名算法为核心,以对优秀dapp地址给与nas奖励为手段的一系列开源的激励机制的集合。简单而言,优秀的dapp将在我们的排名算法中获得较高排名进而获得较高的nas奖励,以此来激励开发者。 我们期望,DIP能使每一个理性的诚实dapp开发者自发的设计优秀的dapp,正如日常生活常见的评奖活动一样。同时,每一个理性的非诚实的用户无法通过各种作弊手段来骗取nas奖励。


3. 激励协议需要满足的性质
\begin{itemize}
	\item 公开性:链上的dapp激励协议与传统的评奖方式最大的不同在于,所有评分的机制必须是完全公开的,且其中任何统计,计算,评选的过程都是全程可见的。这样就杜绝了传统中心化评奖暗箱操作的可能。同时也不会出现票数统计出错等情况。最后,根据评选结果分配奖励的过程也会保证被执行,不会出现卷钱跑路的情况。
	\item 公平性:这也是任何评选机制所要满足的基本性质。我们期望评分高的dapp是活跃用户所喜欢的且经常被调用的,而评分低的dapp是用户鲜有问津的。
	\item 防女巫攻击:区块链技术的一个重大特点就是一个用户建立新的节点地址代价是很小的。所以一个用户有可能建立多个由他控制的地址,并将他们伪装成多个正常用户来参与评选。一个好的激励协议应当保证每个用户无法通过女巫攻击带来巨大额外收益。
	\item 防收买:由于我们衡量dapp好坏的主要指标是活跃用户调用的次数,一个dapp开发者有可能收买大量用户让他们调用自己的dapp以提高自己的排名从而获得更多奖励。这种作弊方式原则上很难避免,但我们期望激励协议能够让此类收买需要付出的代价变得很高以减少其出现的概率。
\end{itemize}

难点:开发者激励协议设计的难点在于将上述性质同时实现,同时保证一定的高效性与可扩展性。事实上,正如区块链(分布式系统)中存在不可能三角一样(cite),任何激励协议也无法在各方面做到最好。

4. 我们的算法

在本文中,我们给出的算法能够在一定程度上实现上述性质。为了实现公平性,我们把用户调用dapp的次数作为影响评分的主要标准。正如各大排行网站出现的点击榜一样,这样能够保证好的dapp受到了更多关注,并能促进整个系统的活跃性。

为了防止女巫攻击,我们应用了星云指数(NR)本身具有的性质。我们让具有更高NR值的用户拥有更多的投票权。由于NR的设计保证了伪造高NR用户的困难性从而有效防止了用户这方面的女巫攻击。同时,我们最终奖励函数的凹函数性质能保证开发者将他的dapp拆分成多个小dapp不会提升他的收益。

为了防止收买,我们给与将票投给多个dapp的用户更高的投票权。因为被收买的用户往往只将票投给一个人,导致效用较低,这样就增加了收买的代价。

本紫皮书的结构如下:TBA
